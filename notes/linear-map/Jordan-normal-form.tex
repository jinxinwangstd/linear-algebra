\documentclass[onecolumn]{ctexart}
\usepackage[utf8]{inputenc}
\usepackage{amsmath}
\usepackage{amssymb}
\usepackage{amsthm}
\usepackage{mathtools}
\usepackage{geometry}
\usepackage{graphicx}
\usepackage{float}
\usepackage{xcolor}
\usepackage{listings}
\usepackage{indentfirst}
\usepackage{bm}
\usepackage{tikz}
\usetikzlibrary{shapes,arrows}
\geometry{a4paper,scale=0.8}

\newcommand*\textinmath[1]{\thickspace\textnormal{#1}\thickspace}

\newtheorem{definition}{Definition}
\newtheorem{theorem}{Theorem}
\newtheorem{proposition}{Proposition}
\newtheorem{lemma}{Lemma}
\newtheorem{corollary}{Corollary}
\newtheorem{remark}{Remark}
\newtheorem{example}{Example}

\DeclareMathOperator{\Img}{Im}

\title{Notes of "Jordan Normal Form"}
\author{Jinxin Wang}
\date{}

\begin{document}

\maketitle

\section{Overview}
\begin{itemize}
  \item Root subspace
  \begin{itemize}
    \item Rmk: The motivation of defining the root subspace
    \item Def: The root subspace of an eigenvalue of a linear operator
    \item Rmk: The root subspace is an invariant subspace of the linear operator
    \item Thm: A direct sum decomposition
  \end{itemize}
  \item Jordan matrix
  \item Proof of the existence of Jordan basis
  \item Proof of the uniqueness of the Jordan normal form of a linear operator
  \item Find the Jordan normal form of a linear operator and a matrix
  \item Find the Jordan basis
\end{itemize}

\end{document}