\documentclass[onecolumn]{ctexart}
\usepackage[utf8]{inputenc}
\usepackage{amsmath}
\usepackage{amssymb}
\usepackage{amsthm}
\usepackage{mathtools}
\usepackage{geometry}
\usepackage{graphicx}
\usepackage{float}
\usepackage{xcolor}
\usepackage{listings}
\usepackage{indentfirst}
\usepackage{bm}
\usepackage{tikz}
\usetikzlibrary{shapes,arrows}
\geometry{a4paper,scale=0.8}

\newcommand*\textinmath[1]{\thickspace\textnormal{#1}\thickspace}

\newtheorem{definition}{Definition}
\newtheorem{theorem}{Theorem}
\newtheorem{proposition}{Proposition}
\newtheorem{lemma}{Lemma}
\newtheorem{corollary}{Corollary}
\newtheorem{remark}{Remark}
\newtheorem{example}{Example}

\DeclareMathOperator{\tr}{tr}

\title{Notes of "Algebra of Linear Operator"}
\author{Jinxin Wang}
\date{}

\begin{document}

\maketitle

\section{Overview}
\begin{itemize}
  \item Definition and basic properties of linear operators
  \begin{itemize}
    \item Def: Linear operator over a vector space
    \item Rmk: The matrix of a linear operator under a basis
    \item Examples of linear operators
    \begin{itemize}
      \item Eg: The zero operator
      \item Eg: The reflection operator on $\mathbb{R}^2$
      \item Eg: The rotation operator on $\mathbb{R}^2$
      \item Eg: The stretch operation on any vector space
      \item Eg: The differentiation operator $\mathcal{D} = \frac{d}{dt}$ on $K_n\lbrack t \rbrack$
    \end{itemize}
    \item Def: Inversible (degenerated, singular) operators, non-inversible (non-degenerated, \\ non-singular) operators, and the inverse operator of a linear operator
    \item Thm: The relationship between the dimension of the kernel and the dimension of the image of a linear operator
  \end{itemize}
  \item The matrices of a linear operator under different bases
  \begin{itemize}
    \item Rmk: The transformation between the matrices of a linear operator under two different bases
    \item Def: Matrix similarity
    \item Thm: The matrices of a linear operator under different bases are similar
    \item Rmk: The matrices of a linear operator under different bases have the same determinant
    \item Def: The determinant of a linear operator
    \item Def: The trace of a square matrix
    \item Rmk: The matrices of a linear operator under different bases have the same trace
    \item Def: The trace of a linear operator
  \end{itemize}
  \item Matrix similarity and its application
  \begin{itemize}
    \item Prop: Matrix similarity is a equivalence relation
    \item Rmk: Our interest in the equivalence classes of matrix similarity and its application
    \item Eg: Find the terms of Fibonacci sequence through the exponentiation of matrices
    \item Rmk: Explore the equivalence classes of matrix similarity by changing the basis of the space
  \end{itemize}
  \item $\mathcal{L}(V)$ as an algebraic structure
  \begin{itemize}
    \item Nta: $\mathcal{L}(V)$
    \item Rmk: $\mathcal{L}(V)$ is a vector space with multiplication
    \item Def: An algebra over a field
    \item Examples of algebras over a field
    \begin{itemize}
      \item $M_n(K)$
      \item $K\lbrack x \rbrack$
      \item An field is an algebra over itself
    \end{itemize}
    \item Rmk: $\sigma: \mathcal{L}(V) \to M_n(K)$ is an (algebra) isomorphism
    \item Def: The determinant function and the trace function on $\mathcal{L}(V)$
    \item Rmk: Some operational properties of the determinant function on $\mathcal{L}(V)$
    \item Rmk: Some operational properties of the trace function on $\mathcal{L}(V)$
  \end{itemize}
  \item The generated subalgebra of a linear operator
  \begin{itemize}
    \item Def: The minimal polynomial of a linear operator
    \item Thm: The existence of the minimal polynomial and its degree
    \item Thm: A necessary and sufficient condition for a linear operator to be inversible in terms of the minimal polynomial
    \item Thm: Every polynomial annihilating a linear operator is divisible by the minimal polynomial of it
    \item Cor: The minimal polynomial of a linear operator is unique
    \item Def: A linear operator is nilpotent and its index
    \item Prop: A necessary and sufficient condition for a linear operator is nilpotent in terms of the minimal polynomial
  \end{itemize}
\end{itemize}

\section{Definition and basic properties of linear operators}

\begin{definition}[Linear operator over a vector space]
  
\end{definition}

\begin{remark}[The matrix of a linear operator under a basis]
  
\end{remark}

\begin{example}[The rotation operator on $\mathbb{R}^2$]
  
\end{example}

\begin{example}[The differentiation operator $\mathcal{D} = \frac{d}{dt}$ on $K_n\lbrack t \rbrack$]
  
\end{example}

\section{The matrices of a linear operator under different basis}

\section{Matrix similarity}

\begin{example}[Find the terms of Fibonacci sequence through the exponentiation of matrices]
  The idea is expressing the recursive relation of the Fibonacci sequence in matrix. Consider the pair $(a_n, a_{n+1})$ and let
  $A = \begin{pmatrix}
    0 & 1 \\
    1 & 1 \\
  \end{pmatrix}$, then 
  \[
    \begin{pmatrix}
      a_n \\
      a_{n+1} \\
    \end{pmatrix} = 
    \begin{pmatrix}
      0 & 1 \\
      1 & 1 \\
    \end{pmatrix}
    \begin{pmatrix}
      a_{n-1} \\
      a_n \\
    \end{pmatrix} = A
    \begin{pmatrix}
      a_{n-1} \\
      a_n \\
    \end{pmatrix} = A^n
    \begin{pmatrix}
      a_0 \\
      a_1 \\
    \end{pmatrix}
  \]
\end{example}

\section{$\mathcal{L}(V)$ as an algebraic structure}

\begin{remark}[Some operational properties of the trace function on $\mathcal{L}(V)$]
  Let $V$ be a vector space over a field $K$. Suppose $\mathcal{A}, \mathcal{B} \in \mathcal{L}(V)$ and $\lambda, \mu \in K$. The trace function on $\mathcal{L}(V)$ has the following operational properties:
  \begin{description}
    \item[R1] $\tr(\lambda \mathcal{A} + \mu \mathcal{B}) = \lambda \tr(\mathcal{A}) + \mu \tr(\mathcal{B})$
    \item[R2] $\tr(\mathcal{A}\mathcal{B}) = \tr(\mathcal{B}\mathcal{A})$
  \end{description}
\end{remark}

\section{The generated subalgebra of a linear operator}

\end{document}