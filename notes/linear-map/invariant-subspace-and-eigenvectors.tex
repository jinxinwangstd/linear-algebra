\documentclass[onecolumn]{ctexart}
\usepackage[utf8]{inputenc}
\usepackage{amsmath}
\usepackage{amssymb}
\usepackage{amsthm}
\usepackage{mathtools}
\usepackage{geometry}
\usepackage{graphicx}
\usepackage{float}
\usepackage{xcolor}
\usepackage{listings}
\usepackage{indentfirst}
\usepackage{bm}
\usepackage{tikz}
\usetikzlibrary{shapes,arrows}
\geometry{a4paper,scale=0.8}

\newcommand*\textinmath[1]{\thickspace\textnormal{#1}\thickspace}

\newtheorem{definition}{Definition}
\newtheorem{theorem}{Theorem}
\newtheorem{proposition}{Proposition}
\newtheorem{lemma}{Lemma}
\newtheorem{corollary}{Corollary}
\newtheorem{remark}{Remark}
\newtheorem{example}{Example}

\DeclareMathOperator{\Img}{Im}

\title{Notes of "Invariant Subspace and Eigenvectors"}
\author{Jinxin Wang}
\date{}

\begin{document}

\maketitle

\section{Overview}
\begin{itemize}
  \item Invariant subspace
  \begin{itemize}
    \item Def: An invariant subspace for a linear operator
    \begin{itemize}
      \item Rmk: Two trivial invariant subspaces for a linear operator
    \end{itemize}
    \item Examples of invariant subspaces of linear operators
    \begin{itemize}
      \item $\Img \mathcal{A}$ and $\ker \mathcal{A}$ are invariant subspaces of $\mathcal{A}$
      \item $\mathcal{A}\mathcal{B} = \mathcal{B}\mathcal{A}$, then $\Img \mathcal{B}$ and $\ker \mathcal{B}$ are invariant subspaces of $\mathcal{A}$
    \end{itemize}
    \item Def: The restriction of a linear operator over an invariant space
    \item Rmk: Connection between an invariant subspace for a linear operator and a matrix of the operator
    \item Thm: A necessary and sufficient condition for the matrix of a linear operator under a basis has the form of block-wise diagonal matrix
    \item Rmk: Generalization of the above theorem to multiple invariant subspaces for a linear operator
    \item Rmk: Specialization of the above remark to invariant subspaces of dimension $1$
    \item Prop: Properties of invariant subspaces for a linear operator
  \end{itemize}
  \item Eigenvectors and characteristic polynomial
  \begin{itemize}
    \item Def: Eigenvector and eigenvalue of a linear operator
    \item Rmk: Connection between eigenvectors and invariant subspaces
    \item Rmk: The eigenvectors corresponding to the same eigenvalue and the zero vector form a subspace
    \item Def: The eigenspace and the geometric multiplicity of an eigenvalue
    \item Thm: A necessary and sufficient condition for a scalar to be an eigenvalue of a linear operator in terms of the roots of a polynomial
    \item Def: The characteristic polynomial of a linear operator
    \item Rmk: A method to find the roots of the characteristic polynomial of a linear operator
    \item Def: The characteristic polynomial of a square matrix
    \item Thm: The characteristic polynomials of similar square matrices are the same
    \item Rmk: The existence of eigenvalues and eigenvectors of a linear operator on a general field and on an algebraically closed field
    \item Study characteristic polynomials by invariant subspaces
    \begin{itemize}
      \item Thm: The characteristic polynomial of a linear operator is divisible by the characteristic polynomial of the restriction of the operator over an invariant subspace
      \item Def: The algebraic multiplicity of an eigenvalue
      \item Cor: The geometric multiplicity of an eigenvalue is no greater than its algebraic multiplicity
      \item Thm: The characteristic polynomial of a linear operator of a space that is a direct sum of invariant subspaces
    \end{itemize}
  \end{itemize}
  \item The criterion of diagonalizable linear operators
  \begin{itemize}
    \item Def: Diagonalizable linear operator
    \item Thm: A necessary and sufficient condition for a linear operator to be diagonalizable in terms of eigenvectors
    \item Lma: The eigenspaces of different eigenvalues are linear independent
    \item Def: The spectrum of a linear operator, the spectrum of a square matrix, 谱点 and 单谱点
    \item Thm: A sufficient condition for a linear operator to be diagonalizable in terms of 单谱点
    \item Thm: A necessary and sufficient condition for a linear operator to be diagonalizable in terms of 谱点
  \end{itemize}
  \item The existence of invariant subspaces
  \begin{itemize}
    \item Thm: A linear operator of a space over an algebraically closed field has one-dimensional invariant subspace(s)
    \item Thm: A linear operator of a space over the real number field has one-dimensional or two-dimensional invariant subspace(s)
  \end{itemize}
\end{itemize}

\section{Invariant subspace}
\begin{definition}[An invariant subspace for a linear operator]
  Let $\mathcal{A}$ be a linear operator on a space $V$. A subspace $U$ of $V$ 
  is called an invariant subspace for $\mathcal{A}$ if $\mathcal{A}U \subset U$, 
  where $\mathcal{A}U = \left\{ \mathcal{A}x \mid x \in U \right\}$.
\end{definition}

\begin{definition}[The restriction of a linear operator over an invariant space]
  
\end{definition}

\begin{remark}[Connection between an invariant subspace for a linear operator and a matrix of the operator]
  Let $U$ be an invariant subspace for a linear operator $\mathcal{A}$ on a 
  space $V$. Choose a basis of $U$ and expand it into a basis of $V$ as $\left\{ 
  e_1, e_2, \ldots, e_n \right\}$. Suppose among them $e_{i+1}, e_{i+2}, \ldots, 
  e_j (i < j)$ are the basis of $U$ and $A_U$ is the matrix of the restriction 
  $\mathcal{A}_U$ under the basis $e_{i+1}, e_{i+2}, \ldots, e_j$. Then the 
  matrix of $\mathcal{A}$ under this basis is that
  \[
    A = 
    \begin{pmatrix}
      A_{11} & 0 & A_{13} \\
      A_{21} & A_U & A_{23} \\
      A_{31} & 0 & A_{33} \\
    \end{pmatrix}
  \]
  in which $A_{11}$ is a square matrix of order $i$, and $A_{33}$ is a square matrix of order $n - j$.

  Proof: Since $U$ is an invariant subspace of $\mathcal{A}$, for any of $e_{i+1}, 
  e_{i+2}, \ldots, e_j$, $\mathcal{A}e_k \in U$ ($k = i+1, i+2, \ldots, j$), and 
  thus a linear combination of $e_{i+1}, e_{i+2}, \ldots, e_j$. Hence the nonzero 
  elements in columns $i+1, i+2, \ldots, j$ can only appear in rows $i+1, i+2, 
  \ldots, j$. More specifically, $\mathcal{A}(e_{i+1}, e_{i+2}, \cdots, e_j) = 
  \mathcal{A}_U(e_{i+1}, e_{i+2}, \cdots, e_j) = (e_{i+1}, e_{i+2}, \cdots, e_j)A_U$, 
  hence the block consisting of columns $i+1, i+2, \ldots, j$ in $A$ is
  $\begin{pmatrix}
    0 \\
    A_U \\
    0 \\
  \end{pmatrix}$ which satisfies
  \[
    \mathcal{A}(e_{i+1}, e_{i+2}, \cdots, e_{j}) = (e_1, e_2, \cdots, e_{i+1}, e_{i+2}, \cdots, e_j, e_{j+1}, \cdots, e_n)
    \begin{pmatrix}
      0 \\
      A_U \\
      0 \\
    \end{pmatrix}
  \]

  Conversely, if the matrix of $\mathcal{A}$ under a basis $\left\{e_1, e_2, 
  \ldots, e_n \right\}$ has the above form, i.e. , then the linear span 
  $\left\langle e_{i+1}, e_{i+2}, \ldots, e_j \right\rangle$ is an invariant 
  subspace for $\mathcal{A}$.
\end{remark}

\begin{remark}[Specialization of the above remark to invariant subspaces of dimension $1$]
  If the space $V$ is a direct sum of $n$ invariant subspaces $V_1, V_2, \ldots, V_n$ of dimension $1$, then with a basis that is compatible with these subspaces, the matrix of the linear operator under this basis has the form as
  \[
    \begin{pmatrix}
      a_1 & 0 & 0 & \cdots & 0 \\
      0 & a_2 & 0 & \cdots & 0 \\
      0 & 0 & a_3 & \cdots & 0 \\
      \vdots & \vdots & \vdots & \ddots & \vdots \\
      0 & 0 & 0 & \cdots & a_n \\
    \end{pmatrix}
  \]
  which is a diagonal matrix.

  Conversely, if the matrix of the linear operator under a basis is a diagonal 
  matrix, then the space can be expressed as a direct sum of $n$ invariant 
  subspaces of dimension $1$ for the operator.
\end{remark}

\section{Eigenvectors and characteristic polynomial}

\begin{definition}[Eigenvector and eigenvalue of a linear operator]
  A non-zero vector $v$ in a space $V$ is called an eigenvector of a linear operator $\mathcal{A}: V \to V$ on $V$ if there exists $\lambda \in K$ such that
  \[
    \mathcal{A}v = \lambda v
  \]
  The scalar $\lambda$ is called an eigenvalue of $\mathcal{A}$ that corresponds to the eigenvector $v$.
\end{definition}

\begin{remark}[Connection between eigenvectors and invariant subspaces]
  Every non-zero vector in an invariant subspace of dimension $1$ for a linear 
  operator is an eigenvector of the operator. Conversely, the linear span of an 
  eigenvector of a linear operator is an invariant subspace for the operator.

  Proof: Let $U$ be an invariant subspace of dimension $1$ for a linear operator $\mathcal{A}$ of the space $V$.
  \begin{description}
    \item[P1] For each non-zero vector $v \in U$, according to the definition of 
    an invariant subspace, we have $\mathcal{A}v \in U$. Since $\dim V = 1$, $v$ 
    is also a basis of $V$, and hence $\mathcal{A}v = \lambda v$. Therefore $v$ 
    is an eigenvector of $\mathcal{A}$.
    \item[P2] Given an eigenvector $v$ of $\mathcal{A}$ with corresponding 
    eigenvalue $\lambda$, the linear span $\left\langle v \right\rangle = \left\{ 
    \mu v \mid \mu \in K \right\}$, then $\mathcal{A}\left\langle v \right\rangle 
    = \left\{ \mathcal{A} \mu v \mid \mu \in K\right\} = \left\{ \mu \mathcal{A} 
    v \mid \mu \in K\right\} =  \left\{ \mu \lambda v \mid \mu \in K\right\} \in 
    \left\langle v \right\rangle$. Therefore, $\left\langle v \right\rangle$ is 
    an invariant subspace for $\mathcal{A}$.
  \end{description}
\end{remark}

\begin{definition}[Eigenspace and geometric multiplicity of an eigenvalue]
  The eigenspace of an eigenvalue $\lambda$ of a linear operator $\mathcal{A}$, 
  denoted by $V^{\lambda}$, is defined as the set of vectors $V^{\lambda} = \lbrace 
  v \in V \mid \mathcal{A}v = \lambda v \rbrace$. The dimension of the eigenspace 
  is called the geometric multiplicity of the eigenvalue $\lambda$.
\end{definition}

\section{The criterion of diagonalizable linear operators}

\begin{definition}[Diagonalizable linear operator]
  A linear operator is diagonalizable if the matrix of it under a basis is a diagonal matrix.
\end{definition}
\begin{remark}[A equivalent condition for a linear operator to be diagonalizable in terms of eigenvectors]
  A linear operator $\mathcal{A}$ on a space $V$ is diagonalizable if and only 
  if $V$ has a basis consists of eigenvectors of $\mathcal{A}$.

  Proof: (TODO)
\end{remark}

\section{The existence of invariant subspaces}

\end{document}