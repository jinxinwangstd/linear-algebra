\documentclass[onecolumn]{ctexart}
\usepackage[utf8]{inputenc}
\usepackage{amsmath}
\usepackage{amssymb}
\usepackage{amsthm}
\usepackage{mathtools}
\usepackage{geometry}
\usepackage{graphicx}
\usepackage{float}
\usepackage{xcolor}
\usepackage{listings}
\usepackage{indentfirst}
\usepackage{bm}
\usepackage{tikz}
\usetikzlibrary{shapes,arrows}
\geometry{a4paper,scale=0.8}

\newcommand*\textinmath[1]{\thickspace\textnormal{#1}\thickspace}

\newtheorem{definition}{Definition}
\newtheorem{theorem}{Theorem}
\newtheorem{proposition}{Proposition}
\newtheorem{lemma}{Lemma}
\newtheorem{corollary}{Corollary}
\newtheorem{remark}{Remark}
\newtheorem{example}{Example}

\title{Notes of "Linear Map between Vector Spaces"}
\author{Jinxin Wang}
\date{}

\begin{document}

\maketitle

\section{Overview}
\begin{itemize}
  \item Definition and related concepts
  \begin{itemize}
    \item Def: A linear map from a space on $K$ to the other on $K$
    \item Examples of linear maps
    \begin{itemize}
      \item Eg: A linear form of a vector space
      \item Eg: Rotation on $\mathbb{R}^2$
      \item Eg: The projection from $\mathbb{R}^3$ to its subspace $\left\{ (a, b, 0) \mid a, b \in \mathbb{R} \right\}$
      \item Eg: $C^k(\mathbb{R}) \to \mathbb{R}_{k+1}\left[ x \right]$
      \item Eg: $\mathcal{D}: K\left[ x \right] \to K\left[ x \right], f(x) \mapsto f'(x)$
    \end{itemize}
    \item Def: The kernel and image of a linear map
    \item Rmk: The kernel and image of a linear map are subspaces of the domain and codomain of the map respectively
    \item Thm: The relationship of the dimensions of the kernel, the image and the domain of a linear map
    \item Cor: The dimension of the image of a linear map is no greater than the dimension of the domain of the map
  \end{itemize}
  \item One-to-one correspondence between linear maps and matrices
  \begin{itemize}
    \item Rmk: Generalize some properties of cases of linear maps
    \item Rmk: A linear map from $V$ to $W$ is uniquely identified by the set of values on a basis
    \item Rmk: A linear map from $V$ to $W$ is uniquely identified by a matrix under certain bases of $V$ and $W$
    \item Def: The rank of a linear map
    \item Thm: Two bijections from $\mathcal{L}(V, W)$ to $W^n$ and from $\mathcal{L}(V, W)$ to $M_{m, n}(K)$
  \end{itemize}
  \item $\mathcal{L}(V, W)$ as an algebraic structure
  \begin{itemize}
    \item Nta: $\mathcal{L}(V, W)$ and $Hom(V, W)$
    \item Rmk: $\mathcal{L}(V, W)$ is a vector space and a subspace of $W^V$
    \item Thm: The bijection $\sigma: \mathcal{L}(V, W) \to M_{m, n}(K)$ is a (linear) isomorphism
    \item Cor: The dimension of $\mathcal{L}(V, W)$ is $\dim V \cdot \dim W$
    \item Rmk: The bijection $\sigma: \mathcal{L}(V, W) \to M_{m, n}(K)$ 保持乘法
    \item Thm: $\dim Im(fg) \leq \dim Im(f)$ and $\dim Im(fg) \leq \dim Im(g)$
  \end{itemize}
  \item The matrix of a linear map represents its coordinate transformation
  \begin{itemize}
    \item Rmk: The matrix of a linear map under certain bases of $V$ and $W$ represents the coordinate transformation under the bases of $V$ and $W$
  \end{itemize}
\end{itemize}

\section{Definition and related concepts}

\section{One-to-one correspondence between linear maps and matrices}

\begin{remark}[A linear map from $V$ to $W$ is uniquely identified by the set of values on a basis]
  Let $V$ and $W$ be two vector spaces, and $e_1, e_2, \cdots, e_n$ be a basis 
  of $V$. We use $\mathcal{L}(V, W)$ denote the set of linear operators from $V$ 
  to $W$, and $W^n$ denote the set of $n$-tuple of vectors (e.g. $(\xi_1, \xi_2, 
  \ldots, \xi_n)$) in $W$.

  We claim that given a basis $e_1, e_2, \cdots, e_n$ of $V$ there is a bijection 
  $\sigma: \mathcal{L}(V, W) \to W^n, f \mapsto (\xi_1, \xi_2, \cdots, \xi_n)$ 
  by setting $f(e_i) = \xi_i, i = 1,2,\ldots,n$.

  First, a linear operator $f$ specifies a set of values on $e_1, e_2, \cdots, 
  e_n$, i.e. specifies a set of vectors as $f(e_1), f(e_2), \cdots, f(e_n)$. 
  Different linear vectors have different sets of values on $e_1, e_2, \cdots, 
  e_n$. If two linear operators $f$ and $g$ have the same sets of values on $e_1, 
  e_2, \cdots, e_n$, i.e. $f(e_i) = g(e_i), i = 1, 2, \ldots, n$, then for every 
  vector $\alpha = \Sigma_{i=1}^n a_i e_i$ in $V$, we have $f(\alpha) = f(\Sigma_{i=1}^n 
  a_i e_i) = \Sigma_{i=1}^n a_i f(e_i) = \Sigma_{i=1}^n a_i g(e_i) = g(\Sigma_{i=1}^n 
  a_i e_i) = g(\alpha)$. Hence $f = g$. We can conclude that $sigma$ is a injection.

  Second, given a set of vectors $\xi_1, \xi_2, \cdots, \xi_n$ in $W$, we can always 
  define a linear map $f: V \to W$ by setting $f(e_i) = \xi_i, i = 1,2,\ldots,n$. 
  By the definition of linear map, for every vector $v = \Sigma_{i=1}^n a_i e_i$ 
  in $V$, $f(v) = f(\Sigma_{i=1}^n a_i e_i) = \Sigma_{i=1}^n a_i f(e_i)$ is determined. 
  Hence the linear map $f$ is specified. Therefore, $\sigma$ is a surjection.
\end{remark}

\begin{remark}[A linear map from $V$ to $W$ is uniquely identified by a matrix under certain bases of $V$ and $W$]
  Let $V$ and $W$ be two vector spaces, $e_1, e_2, \cdots, e_n$ be a basis of $V$, 
  and $w_1, w_2, \cdots, w_m$ be a basis of $W$. We use $\mathcal{L}(V, W)$ denote 
  the set of linear operators from $V$ to $W$, and $M_{m, n}(K)$ denote the set 
  of matrices of order $m \times n$.

  We claim that given a basis $e_1, e_2, \cdots, e_n$ of $V$ and a basis $w_1, 
  w_2, \cdots, w_m$ of $W$ there is a bijection $\sigma: \mathcal{L}(V, W) \to 
  M_{m, n}(K), f \mapsto M_f$ in which the $i$-th ($i = 1, 2, \ldots, n$) column 
  of $M_f$ is the coordinate of $f(e_i)$ under the basis $w_1, w_2, \cdots, w_n$.

  With the basis $e_1, e_2, \cdots, e_n$, we have a bijection $\sigma_1: \mathcal{L}(V, W) \to W^n$. If we can find another bijection $\sigma_2: W^n \to M_{m, n}(K)$, then the mapping $\sigma_2 \sigma_1$ is the one we are looking for in the claim.

  With the basis $w_1, w_2, \cdots, w_m$, the bijection $\sigma_2: W_n \to 
  M_{m, n}(K), (\xi_1, \xi_2, \cdots, \xi_n) \mapsto M$ can be constructed by 
  setting $M = (\xi_1, \xi_2, \cdots, \xi_n)$, i.e. the $i$-th ($i = 1,2,\ldots,
  n$) column of $M$ is the coordinate of $\xi_i$ under the basis $w_1, w_2, \cdots, 
  w_m$. It is clear that $\sigma_2$ is a bijection. Therefore, the claim above 
  is proved to be correct.
\end{remark}

\begin{theorem}[Two bijections from $\mathcal{L}(V, W)$ to $W^n$ and from $\mathcal{L}(V, W)$ to $M_{m, n}(K)$]
  
\end{theorem}

\section{$\mathcal{L}(V, W)$ as an algebraic structure}

\section{The matrix of a linear map represents its coordinate transformation}

\end{document}