\documentclass[onecolumn]{ctexart}
\usepackage[utf8]{inputenc}
\usepackage{amsmath}
\usepackage{amssymb}
\usepackage{amsthm}
\usepackage{mathtools}
\usepackage{geometry}
\usepackage{graphicx}
\usepackage{float}
\usepackage{xcolor}
\usepackage{listings}
\usepackage{indentfirst}
\usepackage{bm}
\usepackage{tikz}
\usetikzlibrary{shapes,arrows}
\geometry{a4paper,scale=0.8}

\newcommand*\textinmath[1]{\thickspace\textnormal{#1}\thickspace}

\newtheorem{definition}{Definition}
\newtheorem{theorem}{Theorem}
\newtheorem{proposition}{Proposition}
\newtheorem{lemma}{Lemma}
\newtheorem{corollary}{Corollary}
\newtheorem{remark}{Remark}
\newtheorem{example}{Example}

\title{Notes of "Basis and Dimension"}
\author{Jinxin Wang}
\date{}

\begin{document}

\maketitle

\section{Overview}
\begin{itemize}
  \item Basis and dimension
  \begin{itemize}
    \item Def: A basis and its basis vectors of a vector space
    \begin{itemize}
      \item Rmk: A basis of a vector contains only finite vectors
    \end{itemize}
    \item Examples of a basis and its basis vectors of a vector space
    \begin{itemize}
      \item Eg: The standard basis of the coordinate space
    \end{itemize}
    \item Thm: The uniqueness of the linear representation (the coordinate) of every vector in a space with a basis
    \item Thm: The existence of a (linear) isomorphism between a space of $n$ dimension and $K^n$
    \item Def: A (linear) homomorphism between two vector spaces, injective homomorphism and surjective homomorphism
    \item Examples of (linear) homomorphisms between two vector spaces
    \begin{itemize}
      \item Eg: $\phi: \mathbb{R}^n \to \mathbb{R}^{\mathbb{N}}$
      \item Eg: $\phi: \mathbb{R}^{\mathbb{N}} \to \mathbb{R}^n$
    \end{itemize}
    \item Def: A (linear) isomorphism between two vector spaces
    \begin{itemize}
      \item Rmk: Two vector spaces can form an isomorphism only if they are on the same field.
      \item Rmk: An isomorphic map between two vector spaces maps the zero vector in one space to the one in the other, and it maps a basis of one space to a basis of the other.
      \item Rmk: If $\phi: V \to U$, $\psi: U \to W$ are both isomorphic maps, then $\phi^{-1}: U \to V$ and $\phi \psi: V \to W$ are both isomorphic maps.
    \end{itemize}
    \item Examples of (linear) isomorphism between two vector spaces
    \begin{itemize}
      \item Eg: $\mathbb{C} \simeq \mathbb{R}^2$
      \item Eg: 实斐波那契数列空间 $\simeq \mathbb{R}^2$
      \item Eg: $M_n(K) \simeq K^{n^2}$
    \end{itemize}
  \end{itemize}
  \item Properties of a basis of a vector space
  \begin{itemize}
    \item Prop: The uniqueness of the number of vectors in every basis of a vector space
    \item Def: The dimension of a vector space
    \item Prop: A necessary and sufficient condition of two vector spaces on the same field to be isomorphic
    \item Thm: The linear independence of $n+1$ vectors of a space of dimension $n$ $(n < \infty)$
    \item Thm: Two necessary and sufficent conditions of a set of vectors to be a basis of a space of dimension $n$ $(n < \infty)$
    \item Thm: The possibility of a set of linear independent vectors to become a basis of a space of dimension $n$ $(n < \infty)$
    \item Def: Finite-dimensional spaces and infinite-dimensional spaces
    \begin{itemize}
      \item Examples of finite-dimensional spaces
      \item Examples of infinite-dimensional spaces
    \end{itemize}
    \item Def: A maximal linearly independent subset of a set of vectors
    \item Thm: Existence, uniqueness of the number of vectors, and the linear span of maximal linearly independent subsets of a set in a finite-dimensional space
    \item Def: The rank of a set of vectors
  \end{itemize}
  \item Transition matrix (Change-of-basis matrix)
  \begin{itemize}
    \item Def: A transition matrix from a basis to another
    \begin{itemize}
      \item Rmk: The uniqueness of the transition matrix from a basis to another
    \end{itemize}
    \item Rmk: Transform the coordinate of a vector under a basis of its space to the coordinate under another basis with the transition matrix between the two basis
    \item Thm: The inversibility of the transition matrix between two basis of a vector space and its meaning
    \item Thm: The composition of two transition matrices between three basis of a vector space and its meaning
  \end{itemize}
  \item The order of the vectors in a basis
  \begin{itemize}
    \item Rmk: By default, a basis of a vector space is an ordered set of vectors, and its order is fixed
  \end{itemize}
\end{itemize}

\section{Basis and dimension}

\begin{definition}[A basis and its basis vectors of a vector space]
  A set of vectors $v_1, v_2, \ldots, v_n$ is said to be a basis of a space $V$ if they are linear independent and their linear span is $V$.
\end{definition}

\begin{remark}[A basis of a vector contains only finite vectors]
  A basis of a vector contains only finite vectors.
\end{remark}

\begin{definition}[A (linear) homomorphism between two vector spaces, injective homomorphism and surjective homomorphism]
  Two vector spaces $U$ and $V$ over a field $K$ are homomorphic if there exists a bijective $\phi: V \to U$ such that $\forall a, b \in K$ and $\forall u, v \in V$ we have
  \begin{equation}
    \phi(au + bv) = a \phi(u) + b \phi(v)
  \end{equation}
\end{definition}

\section{Properties of a basis of a vector space}

\begin{theorem}[Two necessary and sufficent conditions of a set of vectors to be a basis of a space of dimension $n$ $(n < \infty)$]
  Let $V$ be a vector space of dimension $n$, then:
  \begin{description}
    \item[T1] If $n$ vectors $v_1, v_2, \ldots, v_n$ in $V$ are linear independent, then they form a basis of $V$.
    \item[T2] If the linear span of $n$ vectors $v_1, v_2, \ldots, v_n$ in $V$ is $V$ itself, then $v_1, v_2, \ldots, v_n$ are linear independent, and thus are a basis of $V$.
  \end{description}
\end{theorem}

\begin{definition}[Finite-dimensional and infinite-dimensional spaces]
  If any finite set of vectors in a space cannot span the space, then the space is called a infinite-dimensional space. Otherwise, it is a finite-dimensional space.
\end{definition}

\begin{definition}[A maximal linearly independent subset of a set of vectors]
  A subset of set of vectors is called a maximal linearly independent subset if the vectors in the subset are linear independent and every other vector in the set is a linear combination of the vectors in the subset.
\end{definition}

\begin{theorem}[Existence, uniqueness of the number of vectors, and the linear span of maximal linearly independent subsets of a set in a finite-dimensional space]
  Let $S$ be a set of vectors in a finite-dimensional space, and it contains non-zero vector, then:
  \begin{description}
    \item[T1] $S$ contains maximal linearly independent subsets.
    \item[T2] Any two maximal linearly independent subsets of $S$ consists of the same number of vectors.
    \item[T3] A maximal linearly independent subset of $S$ is a basis of $\left\langle S \right\rangle $.
  \end{description}
\end{theorem}

\begin{definition}[The rank of a set of vectors]
  The rank of a set of vectors is defined as the dimension of the linear span of the set of vectors.
\end{definition}

\section{Transition matrix (Change-of-basis matrix)}

\section{The order of the vectors in a basis}

\end{document}