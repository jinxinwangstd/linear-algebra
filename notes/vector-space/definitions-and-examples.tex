\documentclass[onecolumn]{ctexart}
\usepackage[utf8]{inputenc}
\usepackage{amsmath}
\usepackage{amssymb}
\usepackage{amsthm}
\usepackage{mathtools}
\usepackage{geometry}
\usepackage{graphicx}
\usepackage{float}
\usepackage{xcolor}
\usepackage{listings}
\usepackage{indentfirst}
\usepackage{bm}
\usepackage{tikz}
\usetikzlibrary{shapes,arrows}
\geometry{a4paper,scale=0.8}

\newcommand*\textinmath[1]{\thickspace\textnormal{#1}\thickspace}

\newtheorem{definition}{Definition}
\newtheorem{theorem}{Theorem}
\newtheorem{proposition}{Proposition}
\newtheorem{lemma}{Lemma}
\newtheorem{corollary}{Corollary}
\newtheorem{remark}{Remark}
\newtheorem{example}{Example}

\title{Notes of "Definitions and Examples"}
\author{Jinxin Wang}
\date{}

\begin{document}

\maketitle

\section{Overview}
\begin{itemize}
  \item Definitions related to vector spaces
  \begin{itemize}
    \item Def: (Left) group action
    \item Def: Vector space (linear space)
    \item Def: 实向量空间和复向量空间
  \end{itemize}
  \item Properties of operations in a vector space
  \begin{itemize}
    \item Prop: Properties of operations in a vector space
  \end{itemize}
  \item Examples of vector spaces
  \begin{itemize}
    \item Eg: Zero vector space
    \item Eg: Vector spaces of numbers, vectors, or matrices
    \begin{itemize}
      \item Eg: 父域向量空间
      \item Eg: Coordinate space
      \item Eg: $M_{m,n}(K)$和$M_{n}(K)$
    \end{itemize}
    \item Eg: Vector spaces of mappings
    \begin{itemize}
      \item Eg: $\lbrace (a_i)_{i \in \mathbb{N}} \mid a_i \in \mathbb{R} \rbrace$
      \item Eg: $\lbrace (a_i)_{i \in \mathbb{N}} \mid a_i \in \mathbb{R}, a_i = a_{i-1} + a_{i-2}, i \geq 2 \rbrace$
      \item Eg: 集合$X$到向量空间$U$的映射空间$U^X$
      \item Eg: The set of real-valued functions defined on $\mathbb{R}$
      \item Eg: The set of real-valued continuous functions $C(X)$ defined on a subset $X \subset \mathbb{R}$
    \end{itemize}
    \item Eg: Vector spaces of polynomials
    \begin{itemize}
      \item Eg: $m$元多项式环$K[x_1, x_2, \ldots, x_m]$ where $K$ is a field
      \item Eg: $m$元多项式环$K[x_1, x_2, \ldots, x_m]$中次数为$n$的齐次多项式全体加上零多项式
      \item Eg: $m$元多项式环$K[x_1, x_2, \ldots, x_m]$中次数不超过$n$的多项式全体
    \end{itemize}
  \end{itemize}
  \item Linear subspaces
  \begin{itemize}
    \item Def: A linear (or vector) subspace of a vector space
    \begin{itemize}
      \item Rmk: Two trivial subspaces of every vector space
    \end{itemize}
    \item Examples of linear subspaces
    \begin{itemize}
      \item Eg: Any line passing through the origin in $\mathbb{E}^2$
      \item Eg: All vectors parallel to a vector in $\mathbb{E}^3$ forms a subspace of $\mathbb{E}^3$
      \item Eg: The set of solution of the linear system $Ax = 0$ where $A \in M_{m,n}(K)$ forms a subspace of $K^n$
      \item Eg: $\mathbb{R}$的子集$X$上的全体可微函数$C^1(X)$是$C(X)$的一个子空间
      \item Eg: 实数域上定义的偶函数全体,周期函数全体,形如$ae^x + be^-x, a,b \in \mathbb{R}$的函数全体都是实数域上的实函数空间的子空间
    \end{itemize}
    \item Prop: The intersection of any collection of subspaces of a space is a subspace of it
  \end{itemize}
\end{itemize}

\section{Definitions related to vector spaces}

\section{Properties of operations in a vector space}

\begin{proposition}
  Let $-v$ denote the inverse element of $v$.
  \begin{description}
    \item[P1] $0v = 0$
    \item[P2] $a0 = 0$
    \item[P3] $(-1)v = -v$
    \item[P4] $a(-v) = -av$
    \item[P5] $a(u - v) = au - av$
    \item[P6] $(a - b)v = av - bv$
    \item[P7] $av = 0 \Rightarrow a = 0 \textinmath{or} v = 0$ 
  \end{description}
\end{proposition}
\begin{proof}
  Just prove a few of them:
  \begin{description}
    \item[P1] 
    \item[P3] 
    \item[P7] 
  \end{description}
\end{proof}

\section{Examples of vector spaces}

\begin{example}[Coordinate space]
  All row vectors $(a_1, a_2, \ldots, a_n)$ of length $n$ whose elements are in the field $K$.

  Same as all column vectors $[a_1, a_2, \ldots, a_n]$ of length $n$ whose elements 
  are in the field $K$.
\end{example}

\begin{example}[Zero vector space]
  The trivial abel group: $\lbrace 0 \rbrace$, with the scalar multiplication 
  with the field $K$ defined as: $a0 = 0$ for each $a \in K$.

  It is clear that the zero vector space is a vector space on any field.
\end{example}

\begin{example}[The vector space of real-valued functions]
  Let $E$ be a set. The set of all real-valued functions defined on $E$ is a vector 
  space on $\mathbb{R}$ with the following addition and scalar multiplication 
  operations:
  \[
    (f + g)(x) = f(x) + g(x)   
  \]
  \[
    (af)(x) = a(f(x))
  \]
  for each $x \in E$ and $a \in \mathbb{R}$.
\end{example}

\begin{example}[The vector space of mappings to a vector space]
  Let $U$ be a vector space on the field $K$. The set of all mappings from a set 
  $X$ to $U$ is a vector space on $K$, denoted $U^X$. The addition and scalar 
  multiplication are defined as:
  \[
    (f + g)(x) = f(x) + g(x)
  \]
  \[
    (af)(x) = a(f(x))
  \]
  Since $f(x), g(x) \in U$ and $a \in K$, it holds that $f(x) + g(x) \in U$ and 
  $a(f(x)) \in U$ given that $U$ is a vector space on $K$. Therefore, $U^X$ is a 
  vector space on $K$ is guaranteed by that $U$ is a vector space on $K$.

  Especially, $K^X$ is a vector space on $K$. The above example, which is the 
  vector space of all real-valued functions defined on a set $E$, is actually 
  $\mathbb{R}^E$. Hence, $U^X$ is a generalization of the vector space of 
  real-valued functions.
\end{example}

\begin{example}[]
  
\end{example}

\section{Linear subspaces}

\begin{definition}[A linear (or vector) subspace of a vector space]
  Let $U$ be a subset of a vector space $V$ on a field $K$. $U$ is a linear subspace 
  of $V$ if it is an addition subgroup of $V$, and $\forall k \in K$ and $\forall 
  u \in U$ it holds that $ku \in U$.
\end{definition}

\end{document}