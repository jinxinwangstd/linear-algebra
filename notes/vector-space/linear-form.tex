\documentclass[onecolumn]{ctexart}
\usepackage[utf8]{inputenc}
\usepackage{amsmath}
\usepackage{amssymb}
\usepackage{amsthm}
\usepackage{mathtools}
\usepackage{geometry}
\usepackage{graphicx}
\usepackage{float}
\usepackage{xcolor}
\usepackage{listings}
\usepackage{indentfirst}
\usepackage{bm}
\usepackage{tikz}
\usetikzlibrary{shapes,arrows}
\geometry{a4paper,scale=0.8}

\newcommand*\textinmath[1]{\thickspace\textnormal{#1}\thickspace}

\newtheorem{definition}{Definition}
\newtheorem{theorem}{Theorem}
\newtheorem{proposition}{Proposition}
\newtheorem{lemma}{Lemma}
\newtheorem{corollary}{Corollary}
\newtheorem{remark}{Remark}
\newtheorem{example}{Example}

\title{Notes of "Linear Form"}
\author{Jinxin Wang}
\date{}

\begin{document}

\maketitle

\section{Overview}
\begin{itemize}
  \item Linear form, dual space and dual basis of a vector space
  \begin{itemize}
    \item Def: A linear form (or linear functional or covector) of a vector space
    \item Examples of linear forms of vector spaces
    \begin{itemize}
      \item Eg: $f(x_1, x_2, \cdots, x_n) = \Sigma_{i=1}^n a_ix_i$ is a linear map from $K^n$ to $K$
      \item Eg: $\int_a^b f(x)dx$ is a linear map from the space of all continuous functions on $\left[ a, b \right]$ to $\mathbb{R}$
      \item Eg: $Tr: M_n(\mathbb{R}) \to \mathbb{R}, M \mapsto Tr(M)$ is a linear map from $M_n(\mathbb{R})$ to $\mathbb{R}$
    \end{itemize}
    \item Rmk: A necessary and sufficient condition to determine a linear form of a space
    \item Rmk: The set of linear forms from a vector space $V$ to its field $K$ is a subspace of $K^V$
    \item Def: The dual space of a vector space
    \item Rmk: A basis of the dual space of a finite-dimensional space
    \item Thm: The dimension of the dual space of a finite-dimensional space and the dual basis of a basis of the space
    \item Examples of dual space and dual basis
    \begin{itemize}
      \item Eg: The dual space of $P_n = \left\langle 1, t, t^2, \ldots, t^{n-1} \right\rangle$ and the dual basis of the basis $1, t, t^2, \ldots, t^{n-1}$ of the space $P_n$
      \item Eg: The dual space of $V = \left\langle \sin t, \sin 2t, \sin 3t, \ldots, \sin nt \right\rangle$ and the dual basis of the basis $\sin t, \sin 2t, \sin 3t, \ldots, \sin nt$ of the space $V$
    \end{itemize}
  \end{itemize}
  \item A natural isomorphism from a space to the dual space of its dual space
  \begin{itemize}
    \item Rmk: The motivation of examining $V^{**}$
    \item Thm: The map $f: x \mapsto f_x$ where $f_x: V^* \to K, \alpha \mapsto \alpha(x)$ is a (linear) isomorphism from $V$ to $V^{**}$
    \begin{itemize}
      \item Rmk: The (linear) isomorphism $f: x \mapsto f_x$ is natural
      \item Rmk: The symmetry between $V$ and $V^*$ through the isomorphism from $V$ to $V^{**}$
    \end{itemize}
    \item Cor: The one-to-one correspondence between basis of $V$ and basis of $V^*$
    \item Rmk: The conclusion about the equality of the dimension of a finite-dimensional space and the dimension of its dual space doesn't hold for infinite-dimensional spaces and the counterexample of $K\left\lvert x \right\rvert$
  \end{itemize}
  \item Application of linear form: determine the rank of a set of vector in a space
  \begin{itemize}
    \item Lma: The determinant of $n$ linear forms on $n$ linear dependent vectors in a space is $0$
    \item Lma: A necessary and sufficient condition for a set of vectors to be linear independent in terms of the determinant of a basis of $V^*$ on those vectors
    \item Thm: The rank of a set of vectors in a space $V$ is the same as the rank of the matrix consists of the basis of $V^*$ on those vectors
  \end{itemize}
  \item A natural correspondence from subspaces of a space to subspaces of its dual space
  \begin{itemize}
    \item Def: The annihilator of a subspace of a space
    \begin{itemize}
      \item Rmk: The annihilator of a subspace of a space is a subspace of its dual space
    \end{itemize}
    \item Thm: The dimension of the annihilator in terms of the dimension of the space and the subspace
    \item Thm: The annihilator of the annihilator of a subspace give rise to the subspace itself under the isomorphism
    \item Cor: The correspondence between subspaces of a space and subspaces of its dual space in terms of annihilators
  \end{itemize}
  \item A geometric interpretation of the solution space of a homogeneous linear system
  \begin{itemize}
    \item Rmk: The generalization of homogeneous linear systems on a vector space
    \item Thm: The dimension of the solution space can be determined by the dimension of the space and the rank of the set of linear forms
    \item Thm: Every subspace of a vector space is the solution space of a homogeneous linear system on the space
  \end{itemize}
\end{itemize}

\section{Linear form, dual space and dual basis of a vector space}

\section{A natural isomorphism from a space to the dual space of its dual space}

\begin{theorem}[The map $f: x \mapsto f_x$ where $f_x: V^* \to K, \alpha \mapsto \alpha(x)$ is a (linear) isomorphism from $V$ to $V^{**}$]
  The map $f: x \mapsto f_x$ where $f_x: V^* \to K, \alpha \mapsto \alpha(x)$ is a (linear) isomorphism from $V$ to $V^{**}$
\end{theorem}
\begin{proof}
  First, we need to prove that $f_x: V^* \to K, \alpha \mapsto \alpha(x)$ is a linear form of $V^*$.
  \[
    \begin{split}
      f_x(a \alpha + b \beta) &= (a \alpha + b \beta)(x) \\
                              &= a(\alpha(x)) + b(\beta(x)) \\
                              &= a(f_x(\alpha)) + b(f_x(\beta))
    \end{split}
  \]
  From this we can confirm that $f: x \mapsto f_x$ is a map from $V$ to $V^{**}$.

  Second, we need to prove that $f$ is a linear map from $V$ to $V^{**}$.

  Third, we need to prove that $f$ is bijective. To prove this, we can prove that $f$ maps a basis of $V$ to a basis of $V^{**}$.
\end{proof}

\section{Application of linear form: determine the rank of a set of vector in a space}

\section{A natural correspondence from subspaces of a space to subspaces of its dual space}

\section{A geometric interpretation of the solution space of a homogeneous linear system}

\end{document}