\documentclass[onecolumn]{ctexart}
\usepackage[utf8]{inputenc}
\usepackage{amsmath}
\usepackage{amssymb}
\usepackage{amsthm}
\usepackage{mathtools}
\usepackage{geometry}
\usepackage{graphicx}
\usepackage{float}
\usepackage{xcolor}
\usepackage{listings}
\usepackage{indentfirst}
\usepackage{bm}
\usepackage{tikz}
\usetikzlibrary{shapes,arrows}
\geometry{a4paper,scale=0.8}

\newcommand*\textinmath[1]{\thickspace\textnormal{#1}\thickspace}

\newtheorem{definition}{Definition}
\newtheorem{theorem}{Theorem}
\newtheorem{proposition}{Proposition}
\newtheorem{lemma}{Lemma}
\newtheorem{corollary}{Corollary}
\newtheorem{remark}{Remark}
\newtheorem{example}{Example}

\title{Notes of "Linear Relation between Vectors"}
\author{Jinxin Wang}
\date{}

\begin{document}

\maketitle

\section{Overview}
\begin{itemize}
  \item Linear combination and linear span
  \begin{itemize}
    \item Def: A linear combination of a set of vectors
    \item Def: The linear span of a set of vectors
    \begin{itemize}
      \item Rmk: The linear span of a set of vectors is a subspace of the space to which the set of vectors belongs
      \item Rmk: The linear span of a set of vectors is the minimal subspace containing the set of vectors
    \end{itemize}
  \end{itemize}
  \item Linear dependence and independence
  \begin{itemize}
    \item Def: A set of vectors is linear dependent (independent)
    \item Examples of linear dependent and independent vectors
    \begin{itemize}
      \item Eg:
    \end{itemize}
  \end{itemize}
  \item Properties of linear independent and dependent vectors
  \begin{itemize}
    \item Thm: A few sufficient or necessary and sufficient conditions for a set of vectors to be linear dependent
    \item Thm: The linear dependence of $m$ linear combinations out of $n$ vectors
    \item Thm: Some conclusions about the relation between linear combinations and linear dependence
  \end{itemize}
\end{itemize}

\section{Linear combination and linear span}

\begin{definition}[The linear span of a set of vectors]
  The linear span of a subset $M$ of a vector space $V$ is defined as
  \begin{equation}
    \left\langle M \right\rangle = \left\{ a_1x_1 + a_2x_2 + \cdots + a_kx_k \mid a_1, a_2, \ldots, a_k \in K, x_1, x_2, \ldots, x_k \in M, k \in \mathbb{N}^+ \right\} 
  \end{equation}
\end{definition}

\section{Linear dependence and independence}

\section{Properties of linear independent and dependent vectors}

\begin{theorem}[A few sufficient or necessary and sufficient conditions for a set of vectors to be linear dependent]
  We have the following conditions for a set of vectors to be linear dependent:
  \begin{description}
    \item[T1] A set of vectors containing the zero vector are linear dependent.
    \item[T2] A set of at least 2 vectors are linear dependent if and only if one of the vectors is a linear combination of the rest.
    \item[T3] If a subset is linear dependent, then the whole set is linear dependent.
  \end{description}
\end{theorem}

\begin{theorem}[The linear dependence of $m$ linear combinations out of $n$ vectors]
  Let $u_1, u_2, \ldots, u_m$ are linear combinations of $v_1, v_2, \ldots, v_n$:
  \begin{description}
    \item[T1] If $m > n$, then $u_1, u_2, \ldots, u_m$ are linear dependent.
    \item[T2] If $u_1, u_2, \ldots, u_m$ are linear independent, then $m \leq n$.
  \end{description}
\end{theorem}

\begin{theorem}[Some conclusions about the relation between linear combinations and linear dependence]
  We have the following conclusions about linear combinations of a set of vectors:
  \begin{description}
    \item[T1] If $v_1, v_2, \ldots, v_n$ are linear independent, $v_1, v_2, \ldots, v_n, v$ are linear dependent, then $v$ is a linear combination of $v_1, v_2, \ldots, v_n$.
    \item[T2] If $v_1, v_2, \ldots, v_n$ are linear independent, $v$ is not a linear combination of $v_1, v_2, \ldots, v_n$, then $v_1, v_2, \ldots, \\ v_n, v$ are linear independent.
    \item[T3] If $u$ is a linear combination of $v_1, v_2, \ldots, v_n$, and every $v_i$ is a linear combination of $w_1, w_2, \ldots, w_m$, then $u$ is a linear combination of $w_1, w_2, \ldots, w_m$.
  \end{description}
\end{theorem}
\end{document}