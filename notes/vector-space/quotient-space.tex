\documentclass[onecolumn]{ctexart}
\usepackage[utf8]{inputenc}
\usepackage{amsmath}
\usepackage{amssymb}
\usepackage{amsthm}
\usepackage{mathtools}
\usepackage{geometry}
\usepackage{graphicx}
\usepackage{float}
\usepackage{xcolor}
\usepackage{listings}
\usepackage{indentfirst}
\usepackage{bm}
\usepackage{tikz}
\usetikzlibrary{shapes,arrows}
\geometry{a4paper,scale=0.8}

\newcommand*\textinmath[1]{\thickspace\textnormal{#1}\thickspace}

\newtheorem{definition}{Definition}
\newtheorem{theorem}{Theorem}
\newtheorem{proposition}{Proposition}
\newtheorem{lemma}{Lemma}
\newtheorem{corollary}{Corollary}
\newtheorem{remark}{Remark}
\newtheorem{example}{Example}

\title{Notes of "Quotient Space"}
\author{Jinxin Wang}
\date{}

\begin{document}

\maketitle

\section{Overview}
\begin{itemize}
  \item Motivation and definition of quotient space
  \begin{itemize}
    \item Rmk: The motivation of the quotient space is to define a subspace which is naturally isomorphic to any complemented subspace of a subspace
    \item Def: The quotient space of a vector space by a subspace
    \begin{itemize}
      \item Rmk: The operations defined on $V \slash W$ is well-defined, i.e. do not depend on the choice of representatives
    \end{itemize}
    \item Examples of quotient spaces
    \begin{itemize}
      \item Eg: The quotient space of $\mathbb{R}^2$ by the $x$-axis
    \end{itemize}
  \end{itemize}
  \item Properties of quotient space
  \begin{itemize}
    \item Thm: The isomorphism between the quotient space of a vector space by a subspace and a complemented subspace
    \item Cor: The relation between the dimensions of a vector space, a subspace, and the quotient space of the vector space mod by the subspace
  \end{itemize}
\end{itemize}

\section{Motivation and definition of quotient space}

\begin{definition}[The quotient space of a vector space by a subspace]
  Let $V$ be a space over the field $K$ and $U$ be a subspace of $V$. 

  We can define a binary relation in $V$ as follows:
  \[
    x \sim y \Leftrightarrow x - y \in U
  \]
  It is clear that $\sim$ is a equivalence relation.

  The set of all equivalent classes under the above equivalence relation is denoted as $V / U$, and the class where a vector $x$ belongs is denoted as $\bar{x}$. It is clear that
  \[
    \bar{x} = x + U = \left\{ x + u \mid u \in U \right\}
  \]

  We can define the addition and the scalar multiplication with $K$ as follows:
  \[
    \bar{x} + \bar{y} = \bar{x + y}
  \]
  \[
    a\bar{x} = \bar{ax}
  \]
  It is easy to verify that $V/U$ is a vector space over $K$, and it is called the quotient space of $V$ by $U$.
\end{definition}

\section{Properties of quotient space}

\end{document}