\documentclass[onecolumn]{ctexart}
\usepackage[utf8]{inputenc}
\usepackage{amsmath}
\usepackage{amssymb}
\usepackage{amsthm}
\usepackage{mathtools}
\usepackage{geometry}
\usepackage{graphicx}
\usepackage{float}
\usepackage{xcolor}
\usepackage{listings}
\usepackage{indentfirst}
\usepackage{bm}
\usepackage{tikz}
\usetikzlibrary{shapes,arrows}
\geometry{a4paper,scale=0.8}

\newcommand*\textinmath[1]{\thickspace\textnormal{#1}\thickspace}

\newtheorem{definition}{Definition}
\newtheorem{theorem}{Theorem}
\newtheorem{proposition}{Proposition}
\newtheorem{lemma}{Lemma}
\newtheorem{corollary}{Corollary}
\newtheorem{remark}{Remark}
\newtheorem{example}{Example}

\title{Notes of "Quotient Space"}
\author{Jinxin Wang}
\date{}

\begin{document}

\maketitle

\section{Overview}
\begin{itemize}
  \item Rmk: The motivation of the quotient space is to define a subspace which is naturally isomorphic to any complemented subspace of a subspace
  \item Def: The quotient space of a vector space by a subspace
  \begin{itemize}
    \item Rmk: The operations defined on $V \slash W$ is well-defined, i.e. do not depend on the choice of representatives
  \end{itemize}
  \item Thm: The isomorphism between the quotient space of a vector space by a subspace and a complemented subspace
  \item Cor: The relation between the dimensions of a vector space, a subspace, and a complemented subspace of the subspace
\end{itemize}

\end{document}