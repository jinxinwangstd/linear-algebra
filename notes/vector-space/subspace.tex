\documentclass[onecolumn]{ctexart}
\usepackage[utf8]{inputenc}
\usepackage{amsmath}
\usepackage{amssymb}
\usepackage{amsthm}
\usepackage{mathtools}
\usepackage{geometry}
\usepackage{graphicx}
\usepackage{float}
\usepackage{xcolor}
\usepackage{listings}
\usepackage{indentfirst}
\usepackage{bm}
\usepackage{tikz}
\usetikzlibrary{shapes,arrows}
\geometry{a4paper,scale=0.8}

\newcommand*\textinmath[1]{\thickspace\textnormal{#1}\thickspace}

\newtheorem{definition}{Definition}
\newtheorem{theorem}{Theorem}
\newtheorem{proposition}{Proposition}
\newtheorem{lemma}{Lemma}
\newtheorem{corollary}{Corollary}
\newtheorem{remark}{Remark}
\newtheorem{example}{Example}

\title{Notes of "Subspace"}
\author{Jinxin Wang}
\date{}

\begin{document}

\maketitle

\section{Overview}
\begin{itemize}
  \item The sum of two subspaces of a vector space
  \begin{itemize}
    \item Def: The sum of two subspaces of a vector space
    \begin{itemize}
      \item Rmk: The sum of two subspaces is still a subspace
    \end{itemize}
    \item Prop: The existence of a basis of a subspace and the relation between the dimension of a space and the dimension of its subspace
    \item Def: 向量空间的基与子空间相合
    \begin{itemize}
      \item Rmk: A method to find a basis of a vector space 与其一个子空间相合
    \end{itemize}
    \item Thm: The existence of a basis 与两个子空间相合
    \begin{itemize}
      \item Eg: An example that the theorem does not hold for three or more subspaces
    \end{itemize}
    \item Cor: The formula of the basis of the sum of two subspaces
    \item TNta: k-dimensional plane, codimension (余维数) of a subspace, hyperplane (超平面), flag variety (旗簇)
  \end{itemize}
  \item The direct sum of two subspaces of a vector space
  \begin{itemize}
    \item Def: The linear independence and dependence of a class of subspaces
    \item Prop: A necessary and sufficient condition for two subspaces to be linear independent
    \begin{itemize}
      \item Rmk: This proposition cannot be generalized to three or more subspaces
    \end{itemize}
    \item Thm: A necessary and sufficient condition for a class of subspaces to be linear independent in terms of intersection
    \item Thm: A necessary and sufficient condition for a class of subspaces to be linear independent in terms of basis
    \item Thm: A necessary and sufficient condition for a class of subspaces to be linear independent in terms of dimension
    \item Def: The (internal) direct sum of a class of subspaces of a vector space
    \item Thm: The existence of the complement subspace of a subspace
    \item Def: The (external) direct sum of two vector spaces
    \item Rmk: There is no difference in nature between internal direct sums and external direct sums
  \end{itemize}
\end{itemize}

\end{document}